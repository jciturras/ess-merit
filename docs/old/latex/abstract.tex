
\thispagestyle{empty}
\begin{center}
	\section*{{\large Resumen}}
\end{center}
{\setlength{\parindent}{0cm} %Sin sangría
{\normalsize La meritocracia constituye uno de los ideales más representativos de las sociedades modernas de occidente. El mérito, entendido como la combinación de esfuerzos y talentos individuales, constituye el fundamento de una sociedad basada en los principios meritocráticos. La legitimidad de desigualdad social posee un fuerte arraigo en la meritocracia, de modo tal que individuos que se sitúan en estratos más altos de la jerarquía social se conciben a sí mismos como merecedores de su posición. En este sentido ¿son las personas con mayores ingresos, más educadas y con ocupaciones de mayor cualificación, las que perciben más meritocracia? Así también ¿la percepción de sí mismos en la jerarquía social afecta cómo los individuos perciben el funcionamiento de la meritocracia? Esta tesis analizó la relación de características objetivas y subjetivas de estatus social con la percepción de meritocracia en Chile, con el objetivo de determinar en qué medida la experiencia objetiva y las percepciones en torno al estatus social propio, de la familia de origen y de los hijos en el futuro, se relacionan con la percepción de meritocracia. Utilizando datos del Estudio Longitudinal Social de Chile (\textit{n}=2983), se realizó un Análisis Factorial Confirmatorio que evidenció dos dimensiones a través de las cuales puede ser medida la percepción de meritocracia. La primera dimensión guarda relación con la importancia de la ambición, el educarse y el trabajo duro para surgir en la vida. La segunda dimensión se relaciona el grado de acuerdo respecto a si en Chile a las personas son recompensadas por sus esfuerzos y talento. Los resultados de los modelos de ecuaciones estructurales evidenciaron que individuos más educados atribuyen más importancia a los factores individuales para salir adelante, pero a su vez sostienen una visión crítica respecto de las recompensas provenientes del esfuerzo y el talento, mientras que los ingresos y la ocupación no parecen ser relevantes. Respecto de la percepción de estatus social, el estatus propio pierde relevancia al controlar por las medidas de estatus objetivas y subjetivas de estatus, siendo el estatus de los hijos el más relevante, donde individuos con mayores expectativas de logro de estatus para sus hijos, son quienes atribuyen más importancia al mérito individual, y al mismo tiempo, poseen una percepción crítica de las recompensas. Se concluye que las apreciaciones sobre el estatus social en relación al pasado y al futuro, como también el logro educacional, son altamente relevantes en explicar las percepciones sobre la meritocracia en Chile.} 

\vspace{0.5cm}
\textbf{Palabras clave}: Percepción de meritocracia, mérito, estatus social subjetivo, desigualdad social, modelos de ecuaciones estructurales.
}